\section{Root compromise}
This section is divided into two subsections, one describing the current
incident and the second describing actions to be taken, to prevent similar
incidents in the future.

\subsection{The current breach}
While looking at the \texttt{auth.log}, I discovered multiple entries for
failed root login attempts, in total about 192 failed password attempts,
within 4 minutes.

\begin{verbatim}
Mar 22 05:07:39 biomclus1 sshd[12977]: Failed password for root from \
    109.196.143.60 port 52817 ssh2
Mar 22 05:07:39 biomclus1 sshd[12978]: Failed password for root from \
    109.196.143.60 port 52818 ssh2
\end{verbatim}

It seems the attacker used a `simple' brute-force attack. 

As to how the attacker discovered the cluster, I have no idea. A good guess
would be that he scanned for open ports in our range.

\subsection{Preventing future incidents}
In the following several measures to prevent similar attacks are listed, and
then further on described in detail.

\begin{itemize}
\item Stronger passwords
\item Disable root login via SSH
\item Lockout after N failed login attempts
\item Setup DenyHosts
\item Allow only PK login
\item Change SSH port
\item Move cluster to different network
\end{itemize}

A very comprehensive list of security `fixes' for OpenSSH can be found at:
\href{http://www.cyberciti.biz/tips/linux-unix-bsd-openssh-server-best-practices.html}
     {http://www.cyberciti.biz/tips/linux-unix-bsd-openssh-server-best-practices.html}.
The article has inspired some of the above suggested measures.

\subsubsection{Stronger passwords}
The password was guessed within 192 attempts, this could have been prevented by
having a stronger password. If the attacker used a dictionary, this could
suggest that the root password contained words within a dictionary.

To generate stronger passwords, one could use a password generator.

\subsubsection{Disable root login via SSH}
Disallowing the root user to login, would prevent an attacker from logging in
as root. To allow root access to users, {\tt su} or {\tt sudo}, should be used,
this approach has the positive side-effect that actions performed with elevated
privileges can be traced.

This is easily done by setting the \texttt{PermitRootLogin} variable in the 
configuration.

\subsubsection{Change SSH port}
Changing the port that SSHD listens on, will prevent standard script from doing
brute-force attacks. It could also help in preventing the system from being
discovered with simple port scanning.

To change the port, in the configuration one should change the \texttt{Port}
variable to something different from 22.

\subsubsection{Lockout after $N$ failed login attempts}
To delay an attacker when brute-forcing, a mechanism to temporarily lock an 
account after $N$ failed login attempts should be implemented. The mechanism
could work as follows, after 5 failed attempts the account is locked for 2
minutes, then after 15 failed attempts the account should be completely locked.

The module \texttt{pam\_tally} should be installed and configured.

\subsubsection{Setup DenyHosts}
To prevent brute-forcing passwords, one can deny a host trying to login, after
they have failed a specific number of times. Multiple solutions to implement
this exists, ``Fail2ban'' and ``DenyHosts'' are two such.

The most widely used is ``DenyHosts'', which is a python based script to do 
the above described. Setting this up, is described in this article:
\href{http://www.cyberciti.biz/faq/rhel-linux-block-ssh-dictionary-brute-force-attacks/}
     {http://www.cyberciti.biz/faq/rhel-linux-block-ssh-dictionary-brute-force-attacks/}.

\subsubsection{Allow only PK login}
Preventing users from logging in with their password, and only allowing them to
use their Private Key. This prevents an attacker from doing brute-force attacks
using passwords.

This is easily configured through the configuration file, with the
\texttt{PasswordAuthentication} variable.

\subsubsection{Move cluster to different network}
Completely removing the cluster from an internet facing network segment, will
prevent attackers from accessing it - thereby not making brute-force attacks
possible from the outside world. This is in the category of ``extreme''
measures, and I think this would be more of a hassle than actually improving
security.

If this should be implemented, I would suggest placing the cluster behind the 
main firewall, together with the internal servers. To allow outside access one 
could setup a VPN-server, on a separate segment with access only to the cluster,
and then add internal servers as needed.



