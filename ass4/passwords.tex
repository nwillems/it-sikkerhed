\section{Cracking SAM}
In the following a method to crack passwords from a SAM file is first
described, secondly my findings are presented and suggestions on how to improve
password security are made. 

\subsection{Method}
After a visit to Google asking for ``Crack SAM file Linux'', I found an article
describing how to do it with Backtrack2, using ``John the Ripper''. The article
is located here:
\href{http://www.linuxhaxor.net/cracking-windows-admin-pass-with-backtrack2/}
     {http://www.linuxhaxor.net/cracking-windows-admin-pass-with-backtrack2/}

I quickly found the package in my package manager and installed it(Archlinux
has a package called john). After installation I ran the following command 
\texttt{john --format=NT SAM.txt}.

After having john running for 10-15 minutes I got worried and started googling
for online solutions(I though they might have seen a couple of hashes and
cached the results). The results were surprising, at first I found an online
service that would crack a password for 5\$ or so. After looking some more, I
found \href{http://www.md5decrypter.co.uk/ntlm-decrypt.aspx}
    {http://www.md5decrypter.co.uk/ntlm-decrypt.aspx}
, which is free of charge for the first 512 requests.

After inserting the first hash and getting a failure, I was put down a bit.
Until I saw the listing below, someone from Denmark had had a password cracked,
I checked the hashes and they matched, punched it in and saw the results. The
first error was caused by a misunderstanding of the SAM file format.

\subsection{Findings}
The speed of this was quite interesting.

Within a couple of minutes I had both Anni and Lis password, I waited another
10-15 minutes and went to the online service which did it almost instantly.

\begin{table}[h!]
\centering
\begin{tabular}{ l | l }
%
\textbf{Username} & \textbf{Password} \\ \hline
Anni & Anni \\ \hline
Lis & nusser \\ \hline
Peter & bmwX5 \\ \hline
Support & sommer2010 \\ \hline
Administrator & \textit{Not found}
\end{tabular}
\caption{Username and passwords from the SAM file}
\label{tab:userpass}
\end{table}

\subsection{Suggestions for improving password policy}
The password policy currently list the following:

\begin{table}[h!]
\centering
\begin{tabular}{ l | l }
\textbf{Type} & \textbf{Setting} \\ \hline
Password history        & 0 remembered \\ \hline
Max password age        & 42 days \\ \hline
Min password age        & 0 days \\ \hline
Min password length     & 4 characters \\ \hline
Complexity requirements & Disabled \\ \hline
Reversible encryption   & Disabled
\end{tabular}
\caption{Current password policy settings}
\label{tab:passpolicy}
\end{table}

To improve the passwords I would suggest the following changes and actions.

\subsubsection{Password complexity}
This feature of the group policy should be enabled. From Microsoft
Technet\footnote{
\href{http://www.microsoft.com/resources/documentation/windows/xp/all/proddocs/en-us/504.mspx}
{http://www.microsoft.com/resources/documentation/windows/xp/all/proddocs/en-us/504.mspx}}
this would enforce the following:

\begin{itemize}
\item Not contain all or part of the user's account name
\item Be at least six characters in length  
\item Contain characters from three of the following four categories:
\begin{itemize}
    \item English uppercase characters (A through Z)
    \item English lowercase characters (a through z)
    \item Base 10 digits (0 through 9)
    \item Non alphanumeric characters (e.g., !, \$, \#, \%)
\end{itemize}
\end{itemize}

\subsubsection{Password history}
This setting makes the password system remember previous passwords(hopefully
only the hashes), currently this is disabled. It should be enabled with a value
as high as possible, Microsoft says this value can go up to 24 passwords
remembered.

Setting this to 24 means that before users can reuse a password, about 2.5 years has elapsed.

\[
    24\text{ passwords} \times 42\text{ days} = 1008\text{ days} = 
    2.7\text{ years}
\]

\subsubsection{Prevent storing of LANman hashes}
According to a Microsoft support article, one can prevent Active Directory and 
the local SAM database from storing passwords as LAN manager hashes\footnote{
\href{http://support.microsoft.com/kb/299656}
    {http://support.microsoft.com/kb/299656}}.

The approach is quite simple:

\begin{enumerate}
\item In Group Policy, expand: \textbf{Computer Configuration} $\rightarrow$
    \textbf{Windows Settings} $\rightarrow$ \textbf{Security Settings}
    $\rightarrow$ \textbf{Local Policies}, then click \textbf{Security Options}
\item Open up  and enabled the policy named: \textbf{Network security: Do not store LAN
    Manager hash value on next password change}
\end{enumerate}

\subsubsection{Invalidate passwords}
After implementing the above measures, I would invalidate all users passwords,
such that on next logon each user will be prompted to change.

\subsubsection{Implement two/three-factor authorisation}
As a maybe extreme measure, BIOmedix could implement two- or three- factor
authentication, by using security tokens, like the digital key provider for
NemID.

The third factor, could be bio-metrics. This could either be a fingerprint
reader installed at every desk, or using retina scanning with their laptops
built-in web-camera.

\subsection{Password audit aftermath}
\paragraph{Before:} I would talk to corporate management, and ensure that they
are in agreement with the audit. Else I wouldn't do much but start. If the users
are warned that we are doing a password audit, they might change their 
passwords, and then change them back, when the audit is done.

\paragraph{After:} make sure that a report is delivered to the directors, take
the above mentioned actions and measures, and inform the users about possible
outcomes of their weak passwords, and what they can do to prevent it from
happening.
