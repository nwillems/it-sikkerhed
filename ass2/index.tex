\documentclass[paper=a4, fontsize=11pt]{scrartcl} % A4 paper and 11pt font size

\usepackage[T1]{fontenc} % Use 8-bit encoding that has 256 glyphs
\usepackage{fourier} % Use the Adobe Utopia font for the document - comment this line to return to the LaTeX default
\usepackage[english]{babel} % English language/hyphenation
\usepackage{amsmath,amsfonts,amsthm} % Math packages
\usepackage{hyperref}

\usepackage{lipsum} % Used for inserting dummy 'Lorem ipsum' text into the template

\usepackage{sectsty} % Allows customizing section commands
\allsectionsfont{\centering \normalfont\scshape} % Make all sections centered, the default font and small caps

\usepackage{fancyhdr} % Custom headers and footers
\pagestyle{fancyplain} % Makes all pages in the document conform to the custom headers and footers
\fancyhead{} % No page header - if you want one, create it in the same way as the footers below
\fancyfoot[L]{} % Empty left footer
\fancyfoot[C]{} % Empty center footer
\fancyfoot[R]{\thepage} % Page numbering for right footer
\renewcommand{\headrulewidth}{0pt} % Remove header underlines
\renewcommand{\footrulewidth}{0pt} % Remove footer underlines
\setlength{\headheight}{13.6pt} % Customize the height of the header

\numberwithin{equation}{section} % Number equations within sections (i.e. 1.1, 1.2, 2.1, 2.2 instead of 1, 2, 3, 4)
\numberwithin{figure}{section} % Number figures within sections (i.e. 1.1, 1.2, 2.1, 2.2 instead of 1, 2, 3, 4)
\numberwithin{table}{section} % Number tables within sections (i.e. 1.1, 1.2, 2.1, 2.2 instead of 1, 2, 3, 4)

\setlength\parindent{0pt} % Removes all indentation from paragraphs - comment this line for an assignment with lots of text

%----------------------------------------------------------------------------------------
%	TITLE SECTION
%----------------------------------------------------------------------------------------

\newcommand{\horrule}[1]{\rule{\linewidth}{#1}} % Create horizontal rule command with 1 argument of height

\title{	
\normalfont \normalsize 
\textsc{University of Copenhagen} \\ [25pt] % Your university, school and/or department name(s)
\horrule{0.5pt} \\[0.4cm] % Thin top horizontal rule
\huge Current events summary\\ % The assignment title
\Large Breakin at Logica Sweden\\ % Assignment subtitle
\horrule{2pt} \\[0.5cm] % Thick bottom horizontal rule
}

\author{Nicolai Willems} % Your name

\date{\normalsize\today} % Today's date or a custom date

\begin{document}

\maketitle % Print the title

%----------------------------------------------------------------------------------------
%	PROBLEM 1
%----------------------------------------------------------------------------------------

\section{Summary}
The incident described in this summary concerns the recent discovery of a
2-year breakin at Logica, a swedish outsourcing partner. The original article
can be found on
verions2\footnote{
\href{http://www.version2.dk/artikel/hemmelig-rapport-saadan-blev-den-svenske-centraladministration-hacket-51798}
     {http://www.version2.dk/artikel/hemmelig-rapport-saadan-blev-den-svenske-centraladministration-hacket-51798}}.

In conjunction with the arrest and investigation of Gottfrid Svartholm
Warg(GSW), one of the key persons in the "The Pirate Bay"-case, it has surfaced
that he and others have had access to the Logica infrastructure, and some of
their high-profile customers. The breakin started in January 2010 and was
detected beginning of March 2012. 

The attackers succesfully managed to download a confidential database
containing information about all swedish residents, including persons with
hidden indentities. Furthermore they stole tax-information and lists of all
vehichels belonging to the swedish police, along with source-code from IBM and
Logica.

\section{Implications}


\end{document}
