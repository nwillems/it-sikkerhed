\documentclass[paper=a4, fontsize=11pt]{scrartcl} % A4 paper and 11pt font size

\usepackage[T1]{fontenc} % Use 8-bit encoding that has 256 glyphs
\usepackage{fourier} % Use the Adobe Utopia font for the document - comment this line to return to the LaTeX default
\usepackage[english]{babel} % English language/hyphenation
\usepackage{amsmath,amsfonts,amsthm} % Math packages
\usepackage{hyperref}

\usepackage{lipsum} % Used for inserting dummy 'Lorem ipsum' text into the template

\usepackage{sectsty} % Allows customizing section commands
\allsectionsfont{\centering \normalfont\scshape} % Make all sections centered, the default font and small caps

\usepackage{fancyhdr} % Custom headers and footers
\pagestyle{fancyplain} % Makes all pages in the document conform to the custom headers and footers
\fancyhead{} % No page header - if you want one, create it in the same way as the footers below
\fancyfoot[L]{} % Empty left footer
\fancyfoot[C]{} % Empty center footer
\fancyfoot[R]{\thepage} % Page numbering for right footer
\renewcommand{\headrulewidth}{0pt} % Remove header underlines
\renewcommand{\footrulewidth}{0pt} % Remove footer underlines
\setlength{\headheight}{13.6pt} % Customize the height of the header

\numberwithin{equation}{section} % Number equations within sections (i.e. 1.1, 1.2, 2.1, 2.2 instead of 1, 2, 3, 4)
\numberwithin{figure}{section} % Number figures within sections (i.e. 1.1, 1.2, 2.1, 2.2 instead of 1, 2, 3, 4)
\numberwithin{table}{section} % Number tables within sections (i.e. 1.1, 1.2, 2.1, 2.2 instead of 1, 2, 3, 4)

\setlength\parindent{0pt} % Removes all indentation from paragraphs - comment this line for an assignment with lots of text

%----------------------------------------------------------------------------------------
%	TITLE SECTION
%----------------------------------------------------------------------------------------

\newcommand{\horrule}[1]{\rule{\linewidth}{#1}} % Create horizontal rule command with 1 argument of height

\title{	
\normalfont \normalsize 
\textsc{University of Copenhagen} \\ [25pt] % Your university, school and/or department name(s)
\horrule{0.5pt} \\[0.4cm] % Thin top horizontal rule
\huge Current events summary\\ % The assignment title
\Large Breakin at Logica Sweden\\ % Assignment subtitle
\horrule{2pt} \\[0.5cm] % Thick bottom horizontal rule
}

\author{Nicolai Willems} % Your name

\date{\normalsize\today} % Today's date or a custom date

\begin{document}

\maketitle % Print the title

%----------------------------------------------------------------------------------------
%	PROBLEM 1
%----------------------------------------------------------------------------------------

\section{Summary}
The incident described in this summary concerns the recent discovery of a
2-year breakin at Logica, a swedish outsourcing partner. The original article
can be found on
verions2\footnote{
\href{http://www.version2.dk/artikel/hemmelig-rapport-saadan-blev-den-svenske-centraladministration-hacket-51798}
     {http://www.version2.dk/artikel/hemmelig-rapport-saadan-blev-den-svenske-centraladministration-hacket-51798}}.

In conjunction with the arrest and investigation of Gottfrid Svartholm
Warg(GSW), one of the key persons in the "The Pirate Bay"-case, it has surfaced
that he and others have had access to the Logica infrastructure, and some of
their high-profile customers. The breakin started in January 2010 and was
detected beginning of March 2012. 

The attackers succesfully managed to download a confidential database
containing information about all swedish residents, including persons with
hidden indentities. Furthermore they stole tax-information and lists of all
vehichels belonging to the swedish police, along with source-code from IBM and
Logica.

A main entrypoint to the mainframes, was the password database stored in the
system RACF, this database stores usernames and passwords, it is said that it
was downloaded and run through password-cracking tools. This is beleived to
have been succefull, since multiple accounts has logged in during the attack.

\section{The why}
The reasons for breaking in has not been disclosed yet, but a guess is
for profit. The information stolen has high value, and can easily be sold to
interested parties. It is however a bit blurry, wether the intention for the
involved people has been to make money off of the information.

A better guess is interest, the people involved seem to have a profile that
fits this reason, along with the interest in knowing what this data contains. A
show-off mentality could also fit, even though not making a public anouncement
beforehand weighs against this.

A third idea about why this has happened, has been to feed data to an
organisation like Wikileaks. The reason for no discovery/leak yet, is that the
attackers hasn't had time to look through the data and find valuable pieces
yet.

\paragraph{The consequences} experienced in this case, still hasn't been 
discovered. Aside from the data floating around on a privately held PC, no 
physical attacks or leaks has been detected. In conclusion, no immediate
consequences has been seen - but we should follow this closely, the intention
behind this can greatly help in protecting against such attacks.

\section{Implications}
The issues arising from this, spreads from simple loss of confidentiality up to
societal issues. All issues stems from the customer-profile Logica has been
serving, and the data that they store.

\subsection{What if this happens at BIOmedix}
If we at BIOmedix, would have a similar security incident, it could mean loss
of data and thereby money, since we are relying on our knowledge as a good. As
a worst-case scenario, we could have current research leaked to the public or
deleted and thereby lost, setting the process substantially back.

The above are just simple descriptions of the implications. In the long run, if
we had a 2 year undetected attack, it could mean that competitors or even
customers would know our products before we release them, thereby cutting our
profits.

Depending on the nature of the attack, and the intentions behind it. An
attacker could also leak confidential reports about early stages of a product,
giving the public a wrong image of the medicine released. This would at first
destroy our image, but more importantly could be a barrier for patients in need
of the medicine.

\section{Before the attack}
To prevent such attacks, Logica could have been using a stronger password
policy. The following factors are described in the version2 article:

\begin{itemize}
\item Dictionaries
\item defaults
\item weakness
\end{itemize}

Dictionaries are used to check passwords consisting of single or multiple words
listed in a dictionary, this could be simple words like ``correct'',
``battery'', ``horse'' and ``staple''. 

Defaults in this context means passwords that are not changed from what they
originally were. As an example a system comes with the password ``password'', 
this is not changed, therefore leaving the system vulnerable to simple
guessing.

Weakness in passwords, always has its difficulties. We all know the standard
phrases, 8 characters, at least 2 numbers and non-alphabetics, jada jada. This
makes passwords hard to remember. In the current case, the password database
did not distinguish between large and small letters, and did not allow
non-alphabetic characters.

\section{What we should do}
At BIOmedix, I think we should take this incident lightly. Thereby not saying
we should not care, but that we are not in a high-risk group for such attacks.

To close possible holes, I beleive we should dissallow access our mainframes or
the like machines from the internet. Also we should not allow them to
communicate with the internet. This can be done by whitelisting what internal
IP's are allowed to communicate with them.

Our database systems should also be closed down, and only allow access from
inside our network. If we need to have traveling researchers, this could be
opened by a VPN connection.

This could be counter productive, and probably also lead us to more insecure
systems. Having researchers send their research on an open wire to colleagues
for them to put in the databases.

\end{document}
