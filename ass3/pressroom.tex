\section{Pressroom.php}
I have found a couple of issues which I will list and then discuss.

\begin{itemize}
\item Use of \$\_REQUEST
\item Unsanitised input
\item Reading of unintended files
\item Reading of giant files
\end{itemize}

Some of the above mentioned are quite closely related. They are treated 
individually to keep the different flaws separated.

\subsection{Use of \$\_REQUEST}
The use of \texttt{REQUEST} has been discouraged for a while, if not for 
security at least for clarity in code.

The problem here is the way PHP reads values into the \texttt{REQUEST}
variable. It depends on the way PHP is setup with regards to the
\textit{variables\_order}
\footnote{\href{http://www.php.net/manual/en/ini.core.php\#ini.variables-order}
  {http://www.php.net/manual/en/ini.core.php\#ini.variables-order}}
directive.

An attacker could populate a cookie with an unwanted value and have that
inserted each time, since it would be overwritten.

\paragraph{Suggested solution} change to use \texttt{\$\_GET} instead of
\texttt{\$\_REQUEST}

\subsection{Unsanitised input}
The input is never sanitised, it leads to the two below errors, but also has
some unintended ``side-effects''. It opens up for input of unintended strings,
possibly an HTML string.

Since the script uses \texttt{\$\_REQUEST} one could POST data to the script,
allowing them to send much larger fragments than via GET.

\paragraph{Suggested solution} either split the input into three different
parameters, or check that it conforms to a standard format, something like:
Regexp: \texttt{[0-9]\{2\}-[0-9]\{2\}-[0-9]\{2\}}

\subsection{Reading of unintended files}
This relates closely to the unsanitised inputs. The script allows you to put in
a path to a file, and nicely reports back whether the file was found or not.
This allows a possible attacker to read the contents of e.g. index.php or
index.html, a simple read on the source code could reveal database password
etc.

\paragraph{Suggested solution} sanitise inputs with regards to file paths, such
that only allowed files are read, by putting the files in a separate directory
one could check this.

\subsection{Reading of giant files}
This vulnerability also relates to the two above. If an attacker would gain
knowledge of big files accessible by the script, he could ask the script to
read those, and thereby taking up system resources. This could render the
server unresponsive, due to the amount of data that needs to be sent.

\paragraph{Suggested solution} sanitise inputs as describe above.
