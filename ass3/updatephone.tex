\section{Updatephone.c}
I have found a couple of issues which I will list and then discuss.

\begin{itemize}
\item Not checking argc
\item Overflow with user input(use of gets)
\item SQL Sanitising
\item Input sanitising
\end{itemize}

\subsection{Checking argc}
If no argument is specified for the program, it should provide an error message
instead of continuing with invalid inputs.

\paragraph{Suggested solution} check if any user input was given.

\subsection{Overflow (use of gets)}
The use of \texttt{gets} has been discouraged for quite a while. The reason
being simple buffer overflow. If you would input a phone number longer, than
the buffer provided, it will overflow and the program could behave
unintentionally.

\paragraph{Suggested solution} use \texttt{fgets} instead, this function will
check the buffer size and thereby not overflow it.

\subsection{SQL Sanitising}
None of the database code is included, but with the style of programming so
far, it seems like a possible bug. If Bobby from IT changes his phone number to
``123456'); DROP TABLE people; --'' he could possibly delete the entire
database of phone numbers.

\paragraph{Suggested solution} sanitise the input. This could be done simply by
checking that all entered characters are numbers.

\subsection{Input sanitising}
This relates to the string given directly to \texttt{snprintf}, if malicious
Bob where to enter his phone number as ``\%d'' he could possibly lift
information out of the system. This is due to snprintf, when given a format
string, expects more parameters and reads those, in this case from an
unintended space.

\paragraph{Suggested solution} when using formatted output, you could provide a
format string, thereby making the call look like: 

\begin{center}
\texttt{snprintf(phone, sizeof(phone), "\%s", user\_input);}
\end{center}
