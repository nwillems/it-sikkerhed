\documentclass[paper=a4, fontsize=11pt]{scrartcl} % A4 paper and 11pt font size

\usepackage[T1]{fontenc} % Use 8-bit encoding that has 256 glyphs
\usepackage{fourier} % Use the Adobe Utopia font for the document - comment this line to return to the LaTeX default
\usepackage[english]{babel} % English language/hyphenation
\usepackage{amsmath,amsfonts,amsthm} % Math packages
\usepackage{hyperref}

\usepackage{lipsum} % Used for inserting dummy 'Lorem ipsum' text into the template

\usepackage{sectsty} % Allows customizing section commands
\allsectionsfont{\centering \normalfont\scshape} % Make all sections centered, the default font and small caps

\usepackage{fancyhdr} % Custom headers and footers
\pagestyle{fancyplain} % Makes all pages in the document conform to the custom headers and footers
\fancyhead{} % No page header - if you want one, create it in the same way as the footers below
\fancyfoot[L]{} % Empty left footer
\fancyfoot[C]{} % Empty center footer
\fancyfoot[R]{\thepage} % Page numbering for right footer
\renewcommand{\headrulewidth}{0pt} % Remove header underlines
\renewcommand{\footrulewidth}{0pt} % Remove footer underlines
\setlength{\headheight}{13.6pt} % Customize the height of the header

\numberwithin{equation}{section} % Number equations within sections (i.e. 1.1, 1.2, 2.1, 2.2 instead of 1, 2, 3, 4)
\numberwithin{figure}{section} % Number figures within sections (i.e. 1.1, 1.2, 2.1, 2.2 instead of 1, 2, 3, 4)
\numberwithin{table}{section} % Number tables within sections (i.e. 1.1, 1.2, 2.1, 2.2 instead of 1, 2, 3, 4)

\setlength\parindent{0pt} % Removes all indentation from paragraphs - comment this line for an assignment with lots of text

%----------------------------------------------------------------------------------------
%	TITLE SECTION
%----------------------------------------------------------------------------------------

\newcommand{\horrule}[1]{\rule{\linewidth}{#1}} % Create horizontal rule command with 1 argument of height

\title{	
\normalfont \normalsize 
\textsc{University of Copenhagen} \\ [25pt] % Your university, school and/or department name(s)
\horrule{0.5pt} \\[0.4cm] % Thin top horizontal rule
\huge Cloud risk analysis\\ % The assignment title
\horrule{2pt} \\[0.5cm] % Thick bottom horizontal rule
}

\author{Nicolai Willems} % Your name

\date{\normalsize\today} % Today's date or a custom date

\begin{document}

\maketitle % Print the title

%----------------------------------------------------------------------------------------
%	PROBLEM 1
%----------------------------------------------------------------------------------------

This assignment focuses on the risks involved with choosing a cloud provider, in
this particular case Microsoft Office365.

I have divided it into four sections, description of the data stored,
use-cases involved and the personnel needing access, secondly I try to 
approximate the level of security needed for each of the data pieces and 
personnel, the third section matches the needed requirements with the Office365
provided measures of security. Lastly I provide a recommendation on whether
Office365 is suitable or not, for BIOmedix.

\section{Identification of personnel and data}
I have identified three types of information in the organisation:

\begin{description}
\item[Business papers] Describes the processes and flows, together with
    policies in place.
\item[Sales presentations] Slide-shows used by sales reps when presenting for
    possible customers.
\item[Research data] The data produced or gathered by researchers at BIOmedix
\item[Employee records] compasses the data gathered by the HR department. It
    contains, addresses, phone numbers, career plans and performance metrics.
\end{description}

These types of data needs different levels of security, which I will put in
context with the personnel using them. First lets identify the personnel,
involved with handling this data, and their travel habits.

\begin{description}
\item[Common employees] Only works with business processes and management. Works
    from the office and from home.
\item[HR employees] Works with employee records. Works form office and home.
\item[Researchers] Also works at conferences and international research boards.
\item[CEO \& Sales team] Frequently travels and need data on the go
\end{description}

The different teams need access to different information types, in the below
table I will link them together.

\begin{tabular}{r | c | c | c | c}
 & Business papers & Sales slides & research & employee record \\ \hline
Common employees & x &   &   &   \\ \hline
HR employees     & x &   &   & x \\ \hline
Researchers      &   &   & x &   \\ \hline
Sales reps       & x & x &   &   \\ \hline
CEO \& Board     & x & x & x & x \\
\end{tabular}

\section{Security levels}
I have boiled down the above table to four different security levels, with each
of their respective demands and needs for confidentiality. Furthermore I have
tried to quantify the risks involved with loosing such data(All scales from 1
being not dangerous/harmful to 5 being very dangerous/harmfully/costly).

The risk ``matrix'' referenced, is the one presented in the slides from the
lecture held about security management.

\subsection{Personal employee records}
These records are containing possibly damaging information or at least what
danish law describes as being personal, they require a ``special'' level of
trust.

They are however not vital to the company, but could potentially damage the
image of the company if lost or made public.

The risk of this data being stolen, would be a 2. They are of no particular
interest to other companies in the same area of business, but to a non-business
hacker they could be interesting.

The loss of this data could damage up to a 3, reputation and image can be lost.
If it contains more than the ordinary, it could deem very damaging to the
company image.

In the risk ``matrix'' this would end up as something in the yellow zone.

\subsection{Research data}
Research data is critical to the continued growth of business, loss of this data
could render a new product unsaleable, since the data would either be made
public, or sold to competitors.

These data are vital to the company, it is what they sell and base their
operation on.

The risk of this data being stolen is a 5, it is highly likely that competitors
or interested parties would want to get their hands on this kind of data.

The loss of this data could damage up to a 5, the business can suffer
substantial setbacks in product development, or even expose them to a degree of
not being saleable.

In the risk ``matrix'' this would end up as something in the highly red zone.

\subsection{Sales and business}
Business policies and flow descriptions are not crucial to the company, and
are easily recoverable. It can have some political interest for people to
know how a medico works on the inside, but most of this is probably fleshing
out of relevant law material.

Sales presentations however, can contain compromising information with regards
to customer relations, but assuming that BIOmedix operates an open and ``fair''
company, this is not relevant.

The risk of this data being stolen is a 3, it is business description digging
deeper than what is required of the company to publish. Therefore pokey
attackers would interested in making this public.

Loss of this data, scores in at a 2. If completely lost then yes, this can be
quite costly to recover, but a compromise is not crucial for the operations of
business.

In the risk ``matrix'' this would end up as something in the orange zone.

\subsection{All access}
This is more of an access level than an actual security level. It is included
to have a box to put in people needing complete access to all of the businesses
data, such as CEO, boards members and the like.

Needless to say, this is a double 5, hitting in the dark dark red zone of the
risk ``matrix''.

\section{Office365 in context}
In this section, I will explore the security mechanisms Office365 implements. I
will also try to put them in the context of the previously identified security 
levels.


In general Microsoft has very high standards with regards to the data they
store. Which also leads to a high degree of satisfaction for each of the above
described security levels.

Microsoft has released several resources describing the security measures
implemented in their cloud solutions, the following resources has been very
useful:

\begin{itemize}
\item Microsoft Office365 Trust Centre
\href{http://office.microsoft.com/en-us/business/office-365-trust-center-cloud-computing-security-FX103030390.aspx}
     {http://office.microsoft.com/en-us/business/office-365-trust-center-cloud-computing-security-FX103030390.aspx}
\item CSA Registration on Office365
\href{https://cloudsecurityalliance.org/star-registrant/microsoft-office-365/}
     {https://cloudsecurityalliance.org/star-registrant/microsoft-office-365/}
\item QA on Regulatory compliance \href{http://www.microsoft.com/online/legal/v2/?docid=31}
    {http://www.microsoft.com/online/legal/v2/?docid=31}
\item Security controls
\href{http://office.microsoft.com/en-us/business/microsoft-business-security-solutions-FX103045813.aspx}
     {http://office.microsoft.com/en-us/business/microsoft-business-security-solutions-FX103045813.aspx}
\end{itemize}

All the below sub-sections are relevant highlight, alas it is not a
comprehensive list, for a more thorough list, see \textit{Security controls},
for a more in-depth run through of the security surrounding Microsoft' cloud
services checkout the \textit{CSA Registration on Office365}.

\subsection{Data isolation}
Microsoft promises to isolate tenant data from other tenants. Meaning that each
company signed up, will have their data kept isolated. It also means that if
another company is compromised, no access to BIOmedix data is given.

This is crucial for BIOmedix' operation, since it has a high requirement on
isolation for the research data. 

\subsection{Role based access}
User access can be managed through one of several alternative, regular Active
Directory, Azure AD and AD Federation Services. This allows BIOmedix to keep
their current AD with the already implemented Group Policies and current users.

This will minimise the administrative overhead, and the margin for error when
creating new users. It also ensures that if a employee' account is compromised,
no more than what they are allowed is given away to an attacker.

\subsection{Encryption}
Office365 has the possibility to enable Bit-Locker 256-bit encryption for
emails, and using Rights Management Service allows both administrators and
users to selectively encrypt sensitive documents and data.

This further ensures that the BIOmedix information is kept secret from
intruding parties. But in the end, it only allows for a certain security, the
``screen dump'' is still sent only encrypted with SSL over HTTP to the client.

\subsection{Device security}
The Office365 solution also provides the added security to wipe a unit(be it a
PC or smart-phone), if lost or stolen. This feature also provides devices access
restrictions.

This is a vital to BIOmedix' operation of the service, since a large part of
business happens outside the offices, when researchers visits conferences or
sales reps visits customer.

\paragraph{The solution} generally provides high degree of security, and seems
very satisfying to the requirements of BIOmedix. But it is still a Microsoft
provided service, making it subject to all applicable American law, this I see
as a major drawback, for the security of BIOmedix' data.

The involved American applicable law, is mainly the Patriot Act, deeming that a
company based in USA, should deliver data(even that stored in the EU) to the
government if reasonable suspicion about terror is made against the
company, this being done without the company in question knowing about it. I
don't feel that this is reasonable for the data that BIOmedix stores, the risk
of them being investigated due to their area of business is quite high.

\section{Recommendations}
In the end, when evaluating all factors, I can only give recommendations as to 
how Office365 can be used in the BIOmedix organisation.

My recommendation is that for everyone needing to present outside the office,
licenses are bought for Office365. My valuation on the American laws
applicable, does make the Microsoft cloud unusable for the research data, that
BIOmedix treats.

A partial migration could also be done, by having all business documents and 
presentations stored in the cloud, thereby leaving HR records and resource data
to BIOmedix' own security procedures and law applicable in their country of
operation.

I see no scenario where HR records or research data is allowed into the cloud.
Other measures for making this available from home, can be taken. For example a
terminal services solution could be deployed at BIOmedix.

\end{document}
