\documentclass[paper=a4, fontsize=11pt]{scrartcl} % A4 paper and 11pt font size

\usepackage[T1]{fontenc} % Use 8-bit encoding that has 256 glyphs
\usepackage{fourier} % Use the Adobe Utopia font for the document - comment this line to return to the LaTeX default
\usepackage[english]{babel} % English language/hyphenation
\usepackage{amsmath,amsfonts,amsthm} % Math packages
\usepackage{hyperref}

\usepackage{lipsum} % Used for inserting dummy 'Lorem ipsum' text into the template

\usepackage{sectsty} % Allows customizing section commands
\allsectionsfont{\centering \normalfont\scshape} % Make all sections centered, the default font and small caps

\usepackage{fancyhdr} % Custom headers and footers
\pagestyle{fancyplain} % Makes all pages in the document conform to the custom headers and footers
\fancyhead{} % No page header - if you want one, create it in the same way as the footers below
\fancyfoot[L]{} % Empty left footer
\fancyfoot[C]{} % Empty center footer
\fancyfoot[R]{\thepage} % Page numbering for right footer
\renewcommand{\headrulewidth}{0pt} % Remove header underlines
\renewcommand{\footrulewidth}{0pt} % Remove footer underlines
\setlength{\headheight}{13.6pt} % Customize the height of the header

\numberwithin{equation}{section} % Number equations within sections (i.e. 1.1, 1.2, 2.1, 2.2 instead of 1, 2, 3, 4)
\numberwithin{figure}{section} % Number figures within sections (i.e. 1.1, 1.2, 2.1, 2.2 instead of 1, 2, 3, 4)
\numberwithin{table}{section} % Number tables within sections (i.e. 1.1, 1.2, 2.1, 2.2 instead of 1, 2, 3, 4)

\setlength\parindent{0pt} % Removes all indentation from paragraphs - comment this line for an assignment with lots of text

%----------------------------------------------------------------------------------------
%	TITLE SECTION
%----------------------------------------------------------------------------------------

\newcommand{\horrule}[1]{\rule{\linewidth}{#1}} % Create horizontal rule command with 1 argument of height

\title{	
\normalfont \normalsize 
\textsc{University of Copenhagen} \\ [25pt] % Your university, school and/or department name(s)
\horrule{0.5pt} \\[0.4cm] % Thin top horizontal rule
\huge Cloud risk analysis\\ % The assignment title
\horrule{2pt} \\[0.5cm] % Thick bottom horizontal rule
}

\author{Nicolai Willems} % Your name

\date{\normalsize\today} % Today's date or a custom date

\begin{document}

\maketitle % Print the title

%----------------------------------------------------------------------------------------
%	PROBLEM 1
%----------------------------------------------------------------------------------------

This assignment focuses on the risks involved with chosing a cloud provider, in
this particular case Microsoft Office365.

I have divided it into four sections, description of the data stored,
use-cases involved and the personel needing access, secondly I try to 
approximate the level of security needed for each of the data pieces and 
personel, the third section matches the needed requirements with the Office365
provided measures of security. Lastly I provide a recommendation on wether
Office365 is suitable or not, for BIOmedix.

\section{Identification of personel and data}
I have identified three types of information in the organisation:

\begin{description}
\item[Business papers] Describes the processes and flows, together with
    policies in place.
\item[Sales presentations] Slide-shows used by sales reps when presenting for
    possible customers.
\item[Research data] The data produced or gathered by researchers at BIOmedix
\item[Employe records] compasses the data gathered by the HR department. It
    contains, addresses, phone numbers, career plans and performance metrics.
\end{description}

These types of data needs different levels of security, which I will put in
context with the personel using them. First lets identify the personel,
involved with handling this data, and their travel habbits.

\begin{description}
\item[Common employees] Only works with business processes and management. Works
    from the office and from home.
\item[HR employees] Works with employee records. Works form office and home.
\item[Researchers] Also works at conferences and international research boards.
\item[CEO \& Sales team] Frequently travels and need data on the go
\end{description}

The different teams need access to different information types, in the below
table I will link them together.

\begin{tabular}{r | c | c | c | c}
 & Business papers & Sales slides & research & employee record \\ \hline
Common employees & x &   &   &   \\ \hline
HR employess     & x &   &   & x \\ \hline
Researchers      &   &   & x &   \\ \hline
Sales reps       & x & x &   &   \\ \hline
CEO \& Board     & x & x & x & x \\
\end{tabular}

\section{Security levels}
I have boiled down the above table to four different security levels, with each
of their respective demands and needs for confidentiality. Furthermore I have
tried to quantify the risks involved with loosing such data(All scales from 1
being not dangerous/harmfull to 5 being very dangerous/harmfull/costly).

The risk ``matrix'' referenced, is the one presented in the slides from the
lecture held about security management.

\subsection{Personal employee records}
These records are containing possibly damaging information or at least what
danish law describes as being personal, they require a ``special'' level of
trust.

They are however not vital to the company, but could potentially damage the
image of the company if lost or made public.

The risk of this data being stolen, would be a 2. They are of no particular
interest to other companies in the same area of business, but to a non-business
hacker they could be interesting.

The loss of this data could damage up to a 3, reputation and image can be lost.
If it contains more than the ordinary, it could deem very damaging to the
company image.

In the risk ``matrix'' this would end up as something in the yellow zone.

\subsection{Research data}
Research data is critical to the contiued growth of business, loss of this data
could render a new product unsellable, since the data would either be made
public, or sold to competitors.

These data are vital to the company, it is what they sell and base their
operation on.

The risk of this data being stolen is a 5, it is higly likely that competitors
or interested parties would want to get their hands on this kind of data.

The loss of this data could damage up to a 5, the business can suffer
substantial setbacks in product development, or even expose them to a degree of
not being sellable.

In the risk ``matrix'' this would end up as something in the higly red zone.

\subsection{Sales and business}
Business policies and flow descriptions are not crucial to the company, and
are easily recoverable. It can have some political interest for people to
know how a medico works on the inside, but most of this is probably flehsing
out of relevant law material.

Sales presentations however, can contain compromising information with regards
to customer relations, but assuming that BIOmedix operates an open and ``fair''
company, this is not relevant.

\subsection{All access}

% BLA BLA
\section{Office365 in context}


\section{Recommendations}

\end{document}
