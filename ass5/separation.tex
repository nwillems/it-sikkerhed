\section{Separation of networks}
I see multiple reasons to separate the networks, firstly separation of
concerns, secondly the separation of segments provides us with the security
imposed by a possibly restrictive firewall designed in the following section.

\subsection{Separation of Concerns}
The concerns involved with network administration, are in a small company
negligible. But in the context of BIOmedix, we see a full /24 network
reserved for workstations(approximately 256 IPs), sharing these addresses with
the servers minimises the number of connected devices within the workstation
net.

It also frees up resources form the DHCP server within the workstation net.
Addressing at the segment of the internal servers can probably be done in a
static manner, thereby minimising the risk of misconfiguration upon reboots of
servers.

\subsection{Security by ``New Firewall''}
The new firewall imposes some rather strict rules for the devices within the
workstation net, this can open up for a BYOD policy at BIOmedix, but also make
the workstations a very ``unattractive'' place to sit.

It can though limit the damages possible from within the network, f.x. from
an infected workstation or a laptop brought back and forth between work and
home. It also makes breaking in to the workstations a bit harder, the
workstation computers are probably containing current research or even worse
plans for new research projects.

The internal servers segment can also be closed down completely, by setting up
rules in the ``Main Firewall', that disallows any traffic to or from the server
segment, but allows traffic flowing from the workstation segment to both
internet and internal servers. This rule could also be implemented within the
design handed out in \textbf{Exercise4}, but the actual separation of networks,
makes it easier to ``pull the plug'' from either of the networks.
